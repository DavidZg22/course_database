\documentclass[10pt]{article}
\usepackage[utf8]{inputenc}
\usepackage{graphicx}
\usepackage{tikz}
\usepackage[margin=2.54cm]{geometry}
%\usepackage[a4paper, total={6in, 8in}]{geometry}

\usepackage{minted}
% Opciones generales para minted
\setminted[sql]{
  frame=lines,
  framesep=2mm,
  baselinestretch=1.2,
  %bgcolor=lightgray,
  fontsize=\footnotesize,
  linenos
}

\begin{document}

\begin{titlepage}
    \centering
    \vspace*{1cm}
    
    \includegraphics[width=0.8\textwidth]{images/logo.png}
    
    \vspace{0.5cm}
    {\Large Base de datos con SQL SERVER}
    
    \vspace{1.5cm}
    {\LARGE\bfseries Proyecto Final: Score Crediticio}
    
    \vspace{2cm}
    \textbf{Autores:}\\
    David Brandon Zevallos Garay


    \vspace{0.5cm}
    \textbf{Docente:}\\
    Ing. Kevin Rivera Vergaray
    
    \vspace{2cm}
    Lima, Perú\\
    2024
    
    \vfill
    Proyecto Final: Score Crediticio
\end{titlepage}

\tableofcontents
\newpage

\section{Iniciación: Definición de los objetivos y el alcance inicial de la base de datos.}

\subsection{Título del Problema: Gestión de una Institución Educativa}
\subsubsection{Descripción del Problema:}
Una institución educativa necesita gestionar su información de manera efectiva. La base de datos de la institución contiene datos de docentes, especialidades, cursos, asignaciones de docentes a cursos, y matrículas de alumnos. Esta base de datos es crucial para el funcionamiento eficiente de la institución, ya que permite el seguimiento de docentes, alumnos, cursos y matrículas.

\subsubsection{Tablas y campos relevantes:}


\section{Análisis: Diseño Conceptual de la base de datos.}

\section{Diseño: Definición del Modelo Lógico de la Base de Datos.}

\section{Construcción: Desarrollo de las sentencias SQL que permiten la construcción de las tablas y demás estructuras de la Base de Datos en SQL SERVER.}

\begin{minted}{sql}
    CREATE DATABASE trabajo_tienda;
GO

USE tienda;
GO

CREATE TABLE productos(
    id INT PRIMARY KEY IDENTITY(1,1),
    nombre VARCHAR(100) NOT NULL,
    descripcion VARCHAR(500) NOT NULL,
    precio MONEY NOT NULL,
    stock INT NOT NULL
)
GO


CREATE TABLE clientes(
    id INT PRIMARY KEY IDENTITY(1,1),
    nombre VARCHAR(30) NOT NULL,
    apellido VARCHAR(30) NOT NULL,
    direccion VARCHAR(100) NOT NULL,
    telefono VARCHAR(10) NOT NULL
)
GO

CREATE TABLE empleado(
    id INT PRIMARY KEY IDENTITY(1,1),
    nombre VARCHAR(30) NOT NULL,
    apellido VARCHAR(30) NOT NULL,
    direccion VARCHAR(100) NOT NULL,
    telefono VARCHAR(10) NOT NULL
)

CREATE TABLE ventas(
    id INT PRIMARY KEY IDENTITY(1,1),
    id_clientes INT NOT NULL,
    id_vendedor INT NOT NULL,
    fecha DATE NOT NULL,
    cantidad INT NOT NULL,
    total MONEY NOT NULL,
    FOREIGN KEY (id_clientes) REFERENCES clientes(id),
    FOREIGN key (id_vendedor) REFERENCES empleado(id)
)
GO

CREATE TABLE detalles_ventas(
    id INT PRIMARY KEY IDENTITY(1,1),
    id_producto INT NOT NULL,
    id_venta INT NOT NULL,
    cantidad INT NOT NULL,
    precio_unitario MONEY NOT NULL,
    FOREIGN KEY (id_producto) REFERENCES productos(id),
    FOREIGN KEY (id_venta) REFERENCES ventas(id)
)
GO
\end{minted}
    

\section{Producto: Diseño Físico, Script de la Base de datos con inserción de datos (Mínimo 5 registros por tablas independientes o no transaccionales y mínimo 15 registros por cada tabla transaccional).}

\section{Anexos: Evidencias de Consultas, funciones, procedimientos almacenados, vistas, etc. Realizados sobre la base de datos implementada. (5 ejercicios por integrante del grupo, Integrantes individuales (7 ejercicios))}

\end{document}
